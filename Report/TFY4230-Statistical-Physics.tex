% PREAMBLE

%\documentclass[a4paper,10pt,rmp,onecolumn,secnumarabic,numerical,balancelastpage,amsmath,amssymb,hyperref=pdftex,superscriptaddress]{article}

\documentclass[10pt,a4paper,rmp,twocolumn,secnumarabic,numerical,balancelastpage,nofootinbib,hyperref=pdftex,superscriptaddress]{revtex4-2}

\usepackage[T1]{fontenc} % Vise norske tegn
\usepackage{ulem} % for \uuline (double underline)
\usepackage[utf8]{inputenc} % For å kunne skrive norske tegn
\usepackage{babel} % Tilpasning til norsk
\usepackage{graphicx} % For å inkludere grafikk
\usepackage{epstopdf}
\usepackage{caption}
\usepackage{subcaption}
\usepackage{bbm}
%\captionsetup[figure]{name=Figure}
\usepackage{amsmath,amssymb, mathtools} % Ekstra matematikkfunksjoner
\usepackage[dvips,a4paper,margin=2.5cm,bottom=1.5cm]{geometry}
\usepackage{physics} % Use the physics package
\usepackage{dcolumn}% Align table columns on decimal point
\usepackage{geometry} % Required for adjusting page dimensions and margins

\geometry{
	paper=a4paper, % Paper size, change to letterpaper for US letter size
	top=2.5cm, % Top margin
	bottom=2.5cm, % Bottom margin
	left=2.5cm, % Left margin
	right=2.5cm, % Right margin
	headheight=14pt, % Header height
	footskip=1.5cm, % Space from the bottom margin to the baseline of the footer
	headsep=1.2cm, % Space from the top margin to the baseline of the header
	%showframe, % Uncomment to show how the type block is set on the page
}
%\usepackage[backend=bibtex,style=authoryear,autocite=inline]{biblatex}

%\usepackage{authblk} % pakke for author visning
%\renewcommand\Authand{, og } % for å erstatte and med og i author
%\renewcommand\Authands{, og }

%\addbibresource{ref.bib}
%\addtolength\topmargin{-.5\topmargin} %cuts the top margin in half.
%\usepackage{geometry}
%\nocite{*}

\begin{document}

\preprint{APS/123-QED}

\title{Some tediously specific title:\\with Forced Linebreak}% Force line breaks with \\
\thanks{A footnote to the article title}%

\author{Stian S. Johannessen}
 \email{stiansjo@stud.ntnu.no}
 \affiliation{Department of Physics, NTNU, Norway.}
\date{\today}

\begin{abstract}
	Lorem ipsum dolor sit amet, consectetur adipiscing elit, sed do eiusmod tempor incididunt ut labore et dolore magna aliqua. Arcu felis bibendum ut tristique et. Interdum posuere lorem ipsum dolor sit amet consectetur adipiscing elit. Ac orci phasellus egestas tellus rutrum. Cras sed felis eget velit aliquet sagittis id consectetur purus.
\end{abstract}

\maketitle

	\section{one particle}
	
	\begin{figure*}[htb]
		\centering
		\scalebox{0.9}{\input{"./figures/trajectory2000steps.tex"}}
		\label{bg}
		\caption{Trajectory of one particle}
	\end{figure*}

\end{document}